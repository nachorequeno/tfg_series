\chapter{Conclusiones y trabajo futuro}
\label{cha:concl}
Para acabar este trabajo presentamos algunas conclusiones sobre los temas tratados, tanto teóricos como prácticos, y discutimos algunas ideas propicias para la mejora y continuidad del proyecto.

\section{Conclusiones}
Este trabajo ha tenido como objetivo mejorar las capacidades de análisis de las herramientas para la verificación de los sistemas ciberfísicos en tiempo de ejecución. Para ello, hemos utilizado la lógica temporal como formalismo para especificar las propiedades que debe satisfacer el sistema. En particular, hemos usado Signal Temporal Logic (STL), un tipo de lógica temporal enfocada al análisis de señales analógicas que provienen de sensores físicos (p.ej., la velocidad o temperatura).

Entre los resultados más relevantes en este proyecto, hemos logrado extender las capacidades de la lógica temporal STL para expresar propiedades que involucren \textit{tendencias} (derivadas), o \textit{acumulaciones} (integrales). Hemos implementado esos nuevos operadores lógicos en las herramientas software de verificación actuales, principalmente en la herramienta STLEval. Junto con STLEval, nos hemos ocupado también de la biblioteca ParetoLib, una biblioteca de minería que aprende las configuraciones (in)válidas del sistema monitorizado, expresadas como plantillas o especificaciones paramétricas en STL. ParetoLib internamente se conecta con STLEval para la ejecución de los cálculos, por lo que ha recibido una actualización de los binarios y bibliotecas DLL de STLEval que empaqueta de forma conjunta con el resto de la biblioteca para dar soporte a las nuevas operaciones de derivación e integración.

Por último, hemos proporcionado una interfaz de usuario amigable que facilita la interacción tanto con la biblioteca de aprendizaje como indirectamente con la herramienta STLEval. La interfaz gráfica abstrae la complejidad de invocar a STLEval a través de la consola de comandos, o mediante código fuente en Python en el caso de ParetoLib (código fuente~\ref{list:paretolib_example}), tal y como sucedía anteriormente hasta ahora. La interfaz gráfica resume la mayor parte de las opciones de configuración de la herramienta de verificación y del algoritmo de minería.

Finalmente, nos gustaría recalcar el arduo esfuerzo realizado en la documentación e investigación bibliográfica previa para la adquisición de los conocimientos necesarios para la elaboración de los resultados presentados en estas conclusiones.

\section{Trabajo futuro}
Como trabajo futuro, planteamos las siguientes extensiones o mejoras. Por un lado, proponemos incluir más opciones en la interfaz gráfica de usuario que nos permitan configurar parámetros adicionales de las herramientas de verificación y minería y que hasta ahora se han omitido. Por ejemplo, incorporaremos nuevos campos para seleccionar el grado de precisión del aprendizaje, el nivel de paralelismo en los cálculos en ParetoLib u otras elecciones posibles desde el código fuente en Python.

%(variables EPS, DELTA, STEPS)

Por otro lado, actualmente STLEval sólo soporta interpolación constante para rellenar el espacio entre dos puntos de la muestra. Como trabajo futuro, se plantea enriquecer dicha herramienta para soportar nuevos tipos de interpolación (p.ej., lineal, polinomial, splines, etc.). Esto abre la posibilidad de desarrollar nativamente nuevos tipos de operadores lógico-temporales que permitan realizar predicciones sobre el futuro en base a unas señales ya conocidas y recogidas del artefacto que queramos monitorizar (aproximación mediante series de Taylor, análisis estadístico de las trazas u otros métodos de aprendizaje). Un prototipo de nuevo operador lógico-temporal sería el operador \textit{probabilidad} o $\mathbf{Pr}_{\sim \lambda}\varphi$, donde $\sim \in \{<, \leq, \geq, >\}$ y $\lambda \in [0, 1]$.

En siguiente lugar, la implementación de un procesador de lenguaje natural facilitará la escritura de expresiones en STL en un formato más agradable para usuarios no experimentados. En una primera aproximación, simplemente reescribirá las ecuaciones en un formato más legible (p. ej. $F\, G\, (v > 120)$ en lugar de $(F \, (G \, ( > v \, 120))$) para posteriormente permitir en futuras versiones redactar propiedades del estilo \textit{``En el futuro, la propiedad X se cumple''}.

Por último, reimplementar el núcleo de ParetoLib aumentará las prestaciones computacionales de la biblioteca. Python ofrece mecanismos para compilar el código fuente en lugar de interpretarlo, lo que aumentaría el rendimiento de la librería de minería. La mejora sería mínima (ParetoLib llama a STLeval para la mayoría de los cálculos, y STLeval está en C++), pero el incremento del rendimiento al compilar Python nos ayudaría a ahorrar unos pocos milisegundos.