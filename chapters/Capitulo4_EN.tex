\chapter{Conclusions and future work}
\label{cha:concl_en}
\begin{otherlanguage}{british}
To finish this project, we present some conclusions on the topics covered, both theoretical and practical, and we discuss some ideas that are conducive to the improvement and continuity of the project.

\section{Conclusions}
This project has aimed to improve the analysis capabilities of the tools for the verification of cyber-physical systems at runtime. To do this, we have used temporal logic as a formalism to specify the properties that the system must satisfy. In particular, we have used Signal Temporal Logic (STL), a type of temporal logic focused on the analysis of analog signals coming from physical sensors (i.e., speed or temperature).

Among the most relevant results in this project, we have managed to extend the capabilities of STL temporal logic to express properties that involve \textit{trends} (derivatives), or \textit{accumulations} (integrals). We have implemented these new logical operators in the current verification software tools, mainly in the STLEval tool. Along with STLEval, we have also taken care of the ParetoLib library, a mining library that learns (in)valid configurations of the monitored system, expressed as templates or parametric specifications in STL. ParetoLib internally interfaces with STLEval to perform calculations, so it has received an update to the STLEval binaries and DLLs that it packages together with the rest of the library to support the new derivation and integration operations.

Lastly, we have provided a friendly user interface that makes it easy to interact both with the learning library and indirectly with the STLEval tool. The graphical interface abstracts the complexity of invoking STLEval via the command line, or via Python source code in the case of ParetoLib (source code~\ref{list:paretolib_example}), as has been the case up until now. The graphical interface summarizes most of the configuration options of the verification tool and the mining algorithm.

Finally, we would like to emphasize the arduous effort made in the documentation and previous bibliographical research for the acquisition of the necessary knowledge for the elaboration of the results presented in these conclusions.

\section{Future work}
As future work, we propose the following extensions or improvements. On the one hand, we propose to include more options in the graphical user interface that allow us to configure additional parameters of the verification and mining tools and that have been omitted until now. For example, we will incorporate new fields to select the degree of precision of the learning, the level of parallelism in the calculations in ParetoLib or other possible choices from the source code in Python.

%(variables EPS, DELTA, STEPS)

On the other hand, currently STLEval only supports constant interpolation to fill the space between two sample points. As future work, it is proposed to enrich this tool to support new types of interpolation (i.e. linear, polynomial, splines, etc.). This opens the possibility of natively developing new types of logical-temporal operators that allow making predictions about the future based on already known signals collected from the artefact that we want to monitor (Taylor series approximation, statistical analysis of the traces or other learning methods). A prototype of a new logical-temporal operator would be the operator \textit{probability} or $\mathbf{Pr}_{\sim \lambda}\varphi$, where $\sim \in \{<, \leq, \geq, >\}$ and $\lambda \in [0, 1]$.

Next, implementing a natural language processor will make it easier to write expressions in STL in a more pleasing format for inexperienced users. As a first approximation, it will simply rewrite the equations in a more readable format (i.e. $F\, G\, (v > 120)$ instead of $(F \, (G \, ( > v \, 120))$) to later allow in future versions write properties of the style \textit{``In the future, property X is true''}.

Lastly, reimplementing the core of ParetoLib will increase the computational performance of the library. Python offers mechanisms to compile the source code instead of interpreting it, which would increase the performance of the mining library. The improvement would be minimal (ParetoLib calls STLeval for most calculations, and STLeval is in C++), but the performance boost when compiling Python would help us save a few milliseconds.
\end{otherlanguage}